\chapter{额外提醒及注意事项}
主要就是以下几点提醒和一些额外事项:
\section{用工具}
用工具不一定非要是CFD工具/工具书;也可以是GPT-4,github这种;譬如,一些特定的网格如果自己不愿意画也可以在github上找一个相对成熟能用的版本用,改一改参数即可,这样能省许多功夫(当然,如果有瑕疵或者使用上有特殊要求就需要自己在此基础上再做修改了);至于GPT,GPT-4功能目前来说好用很多,免费的new bing和GPT-3.5在解决很多专业相关的问题的时候功能性比较差,且GPT-3.5还会乱写代码;如果没有太大需求还想save money,可以用字节Coze的BOT的API接口(不过需要科学上网);\par
GPT-3.5应付日常的知识性提问问题不大;

\section{查论文}
查书和论文都相对必要(书主要是补一下知识缺漏,找论文就比较有针对性了);一般而言,比如在画网格时候,可以找一下同类问题的论文,看一看论文里的网格是怎样安排的,密度分布如何,效果怎样;又比如,在求解NS方程的过程中,需要给流场添加扰动(噪声),那么噪声相对于流场的背景平均流速的比例应该控制在什么范围比较合理,不会对实验结果产生决定性影响(P.S.在这个问题背景下甚至需要有考虑噪声比例和湍流密度Tu之间的关系的一些中间过程);\par
总之,找论文做参考也是比较重要的一环,适当参考已有的学术成果能少走不少弯路;\par
Web of Science,google学术,Sci-Hub三件套就不多提了,只要不是很重要(重要到某些参数的设置需要在论文里写进参考文献的那种),一般很够用,而且也不会有什么大问题;

\section{问老师}
有问题一定要及时问老师;自己一个人想问题比较容易陷入误区;


\section{Extra Note 1: 环境配置}
在本地或者服务器上,用"vim .bashrc"命令打开预加载环境变量可进行添加or调整;在服务器上略有不同,直接在后方加"module load xxx"即可在打开服务器终端打开的时候预加载环境(编辑后重新打开终端生效),这样就免去了每次都要source一下环境脚本的麻烦;\par

另外,在需要找某些已搭载环境的目录的时候,用"locate xxx"进行搜索;比如,我想要了解intel编译器的mkl库中的lapack库的位置,只需"locate lapack"即可(如果没有找到就不会显示;安装软件的时候如果显示某共享库文件丢失,也可以locate找到没有丢失此库的环境版本进行搭载);


\section{Extra Note 2: VTK/VTU格式浅析}

详细的格式介绍见\href{https://docs.vtk.org/en/latest/design_documents/VTKFileFormats.html}{vtk官网},此处只给出一些用得到的信息;\par


